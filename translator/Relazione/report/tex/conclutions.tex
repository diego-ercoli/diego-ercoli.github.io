\section{Conclusioni}
Sono stati svolti gli esperimenti di traduzione su quattro frasi, come mostrato in Figura \ref{console}.
\begin{figure}[ht]
	\centering 
	\lstinputlisting[label={console},style = javacode, caption ={}]{java/console}  
	\caption{Console Java durante esecuzione del programma}
	\label{console}
\end{figure}
\\ Possiamo quindi concludere dicendo che l'approccio simbolico basato su TransferSintattico ha generato buoni risultati. \\
La Neural Machine Translation oggigiorno è il modello attualmente più in voga date le sue performances, tuttavia necessita di molti dati a disposizione, sotto forma ad esempio di pagine multi-lingua da cui poter apprendere. Gli approcci simbolici hanno il vantaggio di basarsi esclusivamente su regole, quindi si può sulla carta tradurre anche lingue poco diffuse sul Web. Inoltre, nelle reti neurali
le decisioni prese della macchina non sono trasparenti e spesso incomprensibili anche agli occhi degli esperti, mentre negli approcci simbolici i processi di che hanno portato ad un determinato output sono interpretabili e direttamente correggibili in caso di problemi.